\documentclass[english]{article}

\usepackage[english]{babel}
\usepackage[utf8]{inputenc}
\usepackage[T1]{fontenc}
\usepackage{lmodern}
\usepackage{microtype}
\usepackage{amsmath, amssymb}
\usepackage{todonotes}
\usepackage{tikz-cd}

\DeclareMathOperator{\id}{id}
\newcommand{\restrict}[1]{\left.{#1}\right|}

\author{Jon-Magnus Rosenblad}
\date{April 2021}
\title{MAT4520 -- Mandatory Assignment 1}

\begin{document}
\maketitle

\section*{Problem 1}
\subsection*{(a)}
Bilinearity of $\langle\cdot,\cdot\rangle$ follows from bilinearity of $\wedge$ and linearity of $\star$.
By bilinearity it is sufficient to check symmetry for the elements in our basis.
The case of $k=0$ and $k=3$ are trivial, so we proceed with $k=1$.
We see by anti-commutativity and associativity of $\wedge$
that $e_i^\ast \wedge \star e_j^\ast =0$ unless $i=j$.
In all three nontrivial cases we see that $e_i^\ast\wedge\star e_i\ast=e_1^\ast\wedge e_2^\ast\wedge e_3^\ast$,
so $\langle e_i^\ast,e_j^\ast\rangle$ is $1$ for $i=j$ and $0$ otherwise.

\subsection*{(b)}
Bilinearity and the characterization
\begin{equation*}
    \langle e_i^\ast, e_j^\ast\rangle = \begin{cases}
        1 & i=j\\
        0 & i\neq j
    \end{cases}
\end{equation*}
from (a) gives positive-definiteness and that the standard basis is orthonormal.

\subsection*{(c)}
Bilinearity is again induced, and anti-commutativity is induced by the anti-commutativity of $\wedge$ and linearity of $\star$.

\subsection*{(d)}
If we let $e_1^\ast,e_2^\ast,e_3^\ast$ be the positive orientation of the dual basis,
we see that $[e_i^\ast,e_j^\ast]$ completes the right handed system.
By linearity, this property extends to all dual vectors in $A_1(V)$.
By the corresponding property of the cross product we see that 
$[e_i^\ast,e_j^\ast](e_i\times e_j) = 1$,
so in general we have $[u^\ast, v^\ast](u\times v)=1$ or $[u\ast, v\ast] = (u\times v)^\ast$.

\section*{Problem 2}

To use more apparent notation we use $\tau_{\beta\alpha}=\phi_{\beta\alpha}\colon U_{\beta\alpha}\to GL(n,\mathbb R)$,
and $\phi_\alpha\colon U_\alpha\to \mathbb R^m$ with $m=\dim M$ as the chart-morphisms in the atlas.
To follow the notation of the problem description we contract $\tau_{\beta\alpha}(x)=\tau_{\beta\alpha}$
when working over some fixed point, and write $\tau_{\beta\alpha}(u_\alpha) = (\tau_{\beta\alpha}(x))(u_\alpha)$
for the transfer of the vector $u_\alpha$ to the bundle over $U_\beta$.
We will use composition of $(\tau_{\gamma\beta}(x))\circ(\tau_{\beta\alpha}(x))=\tau_{\gamma\beta}\circ\tau_{\beta\alpha}$
as multiplication in $GL(n,\mathbb R)$ without confusion.

\subsection*{(a)}
Reflexivity is immediate from $\tau_{\alpha\beta}\circ\tau_{\beta\alpha}=\id$.
For transitivity we rewrite $\tau_{\gamma\beta}\circ\tau_{\beta\alpha}\circ\tau_{\alpha\gamma}=\id$
to $\tau_{\gamma\beta}\circ\tau_{\beta\alpha}=\tau_{\gamma\alpha}$ 
by multiplying with $\tau_{\gamma\alpha} = \tau_{\alpha\gamma}^{-1}$ on the right.
It follows that 
$\tau_{\beta\alpha}(u_\alpha)=v_\beta$ and $\tau_{\gamma\beta}(v_\beta)=w_\gamma$
implies $(\tau_{\gamma\beta}\circ\tau_{\beta\alpha})(u_\alpha)=\tau_{\gamma\alpha}(u_\alpha)=w_\gamma$
For symmetry, let $\tau_{\beta\alpha}(u_\alpha)=v_\beta$,
so $\tau_{\alpha\beta}(v_\beta)=(\tau_{\alpha\beta}\circ\tau_{\beta\alpha})(u_\alpha)=u_\alpha$.

\subsection*{(b)}
Fix $x\in M$.
Let $\mathbb R^n_\alpha=\{x_\alpha\}\times\mathbb R^n$ be the fiber $x$ under the composite projection $p\circ \iota_\alpha$
where $\iota_\alpha\colon U_\alpha\times\mathbb R^n\to E$ is the natural map induced from the inclusion into the disjoint union.
Because $\{U_\alpha\}$ is a covering of $M$, there is some $U_\alpha$ containing $x$.
Fix $\alpha$.
For every $\beta$ with $U_\beta\ni x$ we have isomorphisms $\tau_{\beta\alpha}\colon \mathbb R^n_\alpha\xrightarrow{\sim}\mathbb R^n_\beta$
as vector spaces.
By the property $\tau_{\alpha\gamma}\circ\tau_{\gamma\beta}\circ\tau_{\beta\alpha}$ the restriction
$\restrict{\iota_\alpha}_{\{x_\alpha\}}\colon \mathbb R^n_\alpha\to p^{-1}(x)$ induces a well defined vector space structure
and becomes an isomorphism.

\subsection*{(c)}
By construction we have a natural covering of $E$ by the charts $U_\alpha\times R^n$ and morphisms
\begin{equation*}
        \tilde \phi_\alpha\colon\left\{\begin{aligned}
            &U_\alpha\times \mathbb R^n 
            &&\to \mathbb R^{m + n}\\
            &[(x_\alpha,v_\alpha)]
            &&\mapsto (\phi_\alpha(x_\alpha), v_\alpha)
        \end{aligned}\right..
\end{equation*}
This gives $E$ the structure of a manifold.
From the previous exercises we know that the change of representative $(x,u_\alpha)\sim (x,v_\beta)$ is related by
$v_\beta=(\tau_{\beta\alpha}(x))(u_\alpha)$.
Our transfer maps are therefore described by
\begin{equation*}
    \tilde \phi_\beta\circ\tilde\phi_\alpha^{-1}\colon (\phi_\alpha(x),u_\alpha)\mapsto (\phi_\beta(x),(\tau_{\beta\alpha}(x))(u_\beta)).
\end{equation*}
Because $\tau_{\beta\alpha}$ is smooth and $\tau_{\beta\alpha}(x)\in GL(n,\mathbb R)$ is linear,
the curried map
\begin{equation*}
    \tilde\tau_{\beta\alpha}\colon \left\{\begin{aligned}
        &\phi_\alpha(U_{\beta\alpha})\times \mathbb R^n
        &&\to \mathbb R^n\\
        &(\phi_\alpha(x),u_\alpha)
        &&\mapsto (\tau_{\beta\alpha}(x))(u_\alpha)
    \end{aligned}
    \right.
\end{equation*}
is smooth when fixing either component, and therefore smooth.
Because $\phi_\beta\circ\phi_\alpha^{-1}$ is smooth as well, $\tilde\phi_\beta\circ\tilde\phi_\alpha^{-1}$ is smooth,
and so the manifold $E$ is smooth.

\subsection*{(d)}
We induce a basis on our tangent bundles $TU_\alpha$ by using the coordinate axes $de_1,\ldots,de_n$.
Our transfer maps $\tau_{\beta\alpha}\colon U_{\beta\alpha}\to GL(n,\mathbb R)$ map
$x$ to the linearization of $\phi_\beta\circ\phi_\alpha^{-1}$ at $\phi_\alpha(x)$.
Let $\psi_{\beta\alpha}=\phi_\beta\circ \phi_\alpha^{-1}$, so $\tau_{\beta\alpha}\colon x\mapsto J_x\psi_{\beta\alpha}$.
We have $\psi_{\alpha\beta}\circ\psi_{\beta\alpha}$ and $\psi_{\alpha\gamma}\circ\psi_{\gamma\beta}\circ\psi_{\beta\alpha}=\id$,
so the linearization $J_x\psi_{\beta\alpha}$ satisfies the same property at any point $x$.
Our collection $\{TU_\alpha\}$ together with the transfer maps $\{\tau_{\beta\alpha}\}$ have the desired structure described above,
and defines an $n$-dimensional vector bundle.

\end{document}
