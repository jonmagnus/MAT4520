\documentclass[english]{article}

\usepackage[english]{babel}
\usepackage[utf8]{inputenc}
\usepackage[T1]{fontenc}
\usepackage{lmodern}
\usepackage{microtype}
\usepackage{amsmath, amssymb}
\usepackage{todonotes}

\DeclareMathOperator{\id}{id}

\author{Jon-Magnus Rosenblad}
\date{April 2021}
\title{MAT4520 -- Mandatory Assignment 1}

\begin{document}
\maketitle

\section*{Problem 1}
\subsection*{(a)}
Bilinearity of $\langle\cdot,\cdot\rangle$ follows from bilinearity of $\wedge$ and linearity of $\star$.
By bilinearity it is sufficient to check symmetry for the elements in our basis.
The case of $k=0$ and $k=3$ are trivial, so we proceed with $k=1$.
We see by anti-commutativity and associativity of $\wedge$
that $e_i^\ast \wedge \star e_j^\ast =0$ unless $i=j$.
In all three nontrivial cases we see that $e_i^\ast\wedge\star e_i\ast=e_1^\ast\wedge e_2^\ast\wedge e_3^\ast$,
so $\langle e_i^\ast,e_j^\ast\rangle$ is $1$ for $i=j$ and $0$ otherwise.

\subsection*{(b)}
Bilinearity and the characterization
\begin{equation*}
    \langle e_i^\ast, e_j^\ast\rangle = \begin{cases}
        1 & i=j\\
        0 & i\neq j
    \end{cases}
\end{equation*}
from (a) gives positive-definiteness and that the standard basis is orthonormal.

\subsection*{(c)}
Bilinearity is again induced, and anti-commutativity is induced by the anti-commutativity of $\wedge$ and linearity of $\star$.

\subsection*{(d)}
\todo{}

\section*{Problem 2}

\subsection*{(a)}
Reflexivity is immediate.
Setting $\alpha=\gamma$ in our triangle relation $\phi{\alpha\gamma}(x)\circ \phi_{\gamma\beta}(x)\circ \phi_{\beta\alpha}(x)=\id$
we get $\phi_{\alpha\beta}(x)\circ \phi_{\beta\alpha}(x) = \id$, and consequently $\phi_{\alpha\beta}(x)=\phi_{\beta\alpha}^{-1}$
as well as the desired symmetry of our relation.
Furthermore we may now rewrite our triangle relation as $\phi_{\alpha\beta}(x)\circ \phi_{\beta\gamma}(x) = \phi_{\alpha\gamma}(x)$ 
and so we have transitivity of our relation, as desired.

\subsection*{(b)}
\todo{Use axioms of manifolds}

\end{document}
